% OWD 2024: IMLO assignment report
%
\documentclass[journal]{IEEEtran}
\usepackage[en-GB]{datetime2}
\usepackage{xcolor}
\usepackage[
    colorlinks,
    linkcolor={red!50!black},
    citecolor={blue!50!black},
    urlcolor={blue!80!black}
]{hyperref}

\newcommand\dotsep{\enspace\textperiodcentered\enspace}
\newcommand\networkperformance{43}

\title{Classification of the \emph{Flowers-102} Dataset with Convolutional Deep
    Neural Networks}
\author{Examination Candidate \#Y3898772%
    \thanks{Manuscript prepared with \LaTeX\ and \texttt{IEEEtran} on \today.}
    \thanks{Submitted in partial fulfilment of the requirements of the 
        \href{https://www.york.ac.uk/students/studying/manage/programmes/%
        module-catalogue/module/COM00026I/2023-24}{\emph{Intelligent Systems:
        Machine Learning and Optimisation}} module assignment at the University
        of York in the 2023/24 academic year.}%
}

\begin{document}
\maketitle
\begin{abstract}
    The classification of data into discrete categories is an ancient problem,
    recently made accessible on extremely large datasets due to substantial
    advances in hardware capability; one such advancement is the introduction of
    graphics processing units (GPUs) in machine learning applications,
    particularly for the training and evaluation of deep neural networks (DNNs).
    This report defines and evaluates such a DNN for the classification of the
    102-category \emph{Flowers} dataset%
    \footnote{\url{https://www.robots.ox.ac.uk/~vgg/data/flowers/102/}} from the
    Visual Geometry Group at the University of Oxford.

    For this purpose, a convolutional deep neural network (CDNN) was
    constructed, using topical understandings of Batch Normalisation (BN)
    techniques for regularisation. A fair 102-sided die would classify an
    arbitrarily chosen image in $\mathbf{0.98}$\% of instances, whereas the
    constructed CDNN achieves $\mathbf{\networkperformance}$\% accuracy on the
    test data split.
\end{abstract}
\section{Introduction}
\section{Method}
\section{Network Architecture}
\section{Results and Evaluation}
\section{Conclusion and Further Work}

\nocite{*} % TODO remove
\bibliographystyle{IEEEtran}
\bibliography{bibliography}
\end{document}

